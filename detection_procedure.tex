\subsection{Detection algorithm}
The purpose of the detection algorithm is to detect potentially too exposed devices. It analyzes the LoRa traffic, keeps track of all the detected DevAddress, and uses a pattern-matching strategy to verify if two collected DevAddress could be associated with the same ED.

\vspace{5mm}

In detail, the detection algorithm uses a vector \(\ C \) to store all the identified patterns. It contains a series of elements \(\ P_{a} \), where \(\ a \) is the current DevAddress associated to an ED, and \(\ P_{a} = \{ s_{1}, ..., s_{n} \} \) is the pattern the device follows. Since different patterns necessarily describe distinct EDs, in \(\ C \), there cannot be two tuples denoting the same device. In other words, let \(\ E \) be the set of all the ED that currently join the network, we have:
\[\ length(C) \subseteq length(E ) \]
The vector \(\ C \) allows checking if a given device, represented by its current DevAddres \(\ a \), already exchanged messages in the network. When the detection algorithm reads a new packet with DevAddress \(\ a \), it observe if the condition \(\ P_{a} \in C \) occurs. If \(\ P_{a} \notin C \) and this happens after the interception of a Join-request message, PIVOT  cannot establish \textit{a priori} the nature of the device associated to \(\ a \). It initiates a pattern-matching process to compute if \(\ a \) belongs to a new device or a re-joined one, trying to achieve this task in the shortest time possible. In this way, in the case of potential matching, the algorithm minimizes the delay in triggering the alert.

\vspace{5mm}

Intercepting a Join-request message is a crucial part of the detection process. It permits PIVOT to comprehend which addresses it needs to investigate and which patterns we should use for the matching procedure. The pattern \(\ P_{a} \notin C \), associated with an unknown address \(\ a \), should \textit{not} be compared with all the other patterns of \(\ C \). We must exclude from our analysis all that pattern related to these two classes of addresses:

\vspace{3mm}
\begin{itemize}
	\item All the addresses collected after the Join request, which have been noticed also before the Join-request. They are associated with devices that never disconnect from the network.
	\item All the other unknown addresses received after the Join-request, since a new device that joins the network has not more than one DevAddr.
\end{itemize}
\vspace{3mm}

The algorithm follows two sequential steps, called \textit{Pre-Join} procedure and \textit{Main} procedure. Figure \ref{fig:pre_main} illustrates the two procedures. Let \(\ t_{0} \) the timestamp of the first packet received by PIVOT after its activation and \(\ t_{f} \) the timestamp of the first Join-request packet. The Pre-Join procedure takes place in the time window \(\ t_{f} - t_{0} \) and the Main procedure starts after the reception of the first Join-Request message, at \(\ t_{f} \). In the next sections, we are going to examine these two steps in detail.

\begin{figure}[H]
    \centering
    \vspace{4mm}
    \includegraphics[width=0.7\linewidth]{images/pivot/pre-main.PNG}
    \caption{Representation of the time windows of the Pre-Join and Main procedures}
    \label{fig:pre_main}
\end{figure}
\vspace{5mm}

\subsubsection{Pre-Join procedure}
The Pre-Join procedure is a necessary step to give to PIVOT an initial view about which EDs are currently operating in the network. Indeed, when the system is turned on and start receiving packets, it has no historical trace of the devices in the network, either of the messages they sent before its activation. This implies that the vector \(\ C \) is initially \textit{empty} and therefore, there are no saved pattern to use in the matching process. Then, the Pre-Join procedure is a initial step through which the first patterns are identified and modeled. They will subsequently be used as a reference in the \textit{Main} procedure.

\vspace{5mm}

In detail, for each packet \(\ p \) received in input, the algorithm reads the DevAddress \(\ a \) and the timestamp \(\ t \). There are two scenarios:

\vspace{3mm}
\begin{enumerate}
	\item \(\ P_{a} \notin  C\). A new pattern \(\ P_{a} \) is \textit{initialized}.
	\item \(\ P_{a} \in C\). The pattern \(\ P_{a} \) is \textit{updated}
\end{enumerate}
\vspace{3mm}

According to the Section \ref{updating}, the initialization of a new pattern consists in the creation of an empty chain \(\ P_{a} = \{\} \) to store together with the DevAddress \(\ a \), the timestamp \(\ t \) and the flag \(\ verified \), set to \textit{False}. In the same way, updating an existing pattern implies calculating the \textit{current} segment \(\ t - t_{f} \) and checking if \(\ s \) is already in the chain.

\vspace{4mm}
\begin{algorithm}
    \caption{Pre-Join procedure}
    \begin{algorithmic}[1]
        \If{$a$ in $C$}
            \State $P \gets C(a)$
            \State $P.update(t)$
        \Else
            \State $P \gets newPattern(a)$
            \State $C(a) \gets P$
        \EndIf
    \end{algorithmic}
\end{algorithm}
\vspace{4mm}

The Pre-Join terminates when PIVOT receives as input the first Join-request. At the end of this phase, the number of DevAddress \(\ a_{1}, a_{2}, ..., a_{n} \) collected matches with the number of devices that sent messages \(\ ED_{1}, ED_{2}, ..., ED_{n} \). According to the LoRaWAN specification, a device \(\ ED_{i} \) modifies the DevAddress \(\ a_{i} \) only when it disconnects from the network and performs an new OOTA. This step ends exactly at the moment the first Join-request arrives. Then, for the entire duration of the Pre-Join procedure there is no \(\ ED \) that changes the associated DevAddress. All the pattern detected and stored in the vector \(\ C \) are in a \textit{one-to-one} relationship with all the collected DevAddress.

\vspace{5mm}

As we have seen, a variable that determines the temporal window of the Pre-Join is the instant in which a device sends a Join-request. It is an uncertain and unpredictable phenomenon that can affect the accuracy of the detected patterns of \(\ C \). Indeed, if the period \(\ \tau \) of a pattern is longer than the duration of the procedure, the algorithm cannot correctly estimate all the segments that make up the chain. The flag \textit{verified} is designed precisely to indicate the completeness of a pattern. Let \(\ V \) the list of patterns with the flag \textit{verified = True}, then \(\ V \subseteq C\). In the best case, at the end of the Pre-Join, the patterns in the vector \(\ C  \) are all complete, allowing the detection algorithm to perform more precise matches.

\vspace{3mm}
\begin{figure}
    \centering
    \includegraphics[width=0.7\linewidth]{images/pivot/missing_pattern.png}
    \caption{An example of pattern with different periods \(\ \tau_{1} = 0.5 \) and \(\ \tau_{2} = 2.3 \). At the end of the Pre-Join procedure, only the first pattern is recognized.}
\label{fig:missing}
\end{figure}
\vspace{3mm}

The example reported in figure \ref{fig:missing} shows two different patterns \(\ P_{1} = \{ s_{1} \} \) and \(\ P_{2} = \{ s_{2}, s_{3} \} \), where \(\ s_{1} = 0.5 \), \(\ s_{2} = 0.33 \) and \(\ s_{3} = 2.0 \). If the duration of the Pre-Join procedure is an hour and three quarters (or \textit{1.75}), the algorithm can't determine all the segments that composes the chain of \(\ P_{2} \). Then to \(\ P_{2} \) is associated the flag \textit{verified = False}. On the contrary, since \(\ P_{1} \) has a smaller period \(\ \tau = 0.5\),  is correctly recognized, the flag \textit{verified} of \(\ P_{1} \) is set to True.
\vspace{5mm}
\begin{mintedbox}{python}
def __main_procedure(self, p):
    
    # gets the Dev Address
    devaddr = p.dev_addr  
    
    # checks if the pattern already exists
    if devaddr in self.confirmed:
        # rest of code

    else:
        # check if the DevAddress has already been received
        if devaddr in self.unconfirmed:
            # rest of code

        else:
            # initializing a new pattern
            self.unconfirmed[devaddr] = Pattern(p.t)
            self.to_analyze[devaddr] = list(self.confirmed.keys())
\end{mintedbox}

\subsection{Metrics for operator}
PIVOT supports the network operator in its routine operations, providing a series of statistical metrics that illustrate the current state of the administered network.

\subsubsection{Number of Joins (NoJ)}
The Number of Joins (NoJ) represents the overall \textit{Join-request} messages intercepted and registered by PIVOT over the time. Generally, in LoRaWAN there is a \textit{peak} of Join-requests only in the starting phase when the operator registers and activates all the devices that will operate in the network. All Join-requests following this peak denote the insertion of new devices or the reconnection of old ones, two uncommon events. A too high value of NoJ might indicate the malfunction of one or more devices that are disconnecting and reconnecting continuously from the network. As demonstrated, sending a Join-request message could involve the expose the addresses of the device to unauthorized parties, then operator should avoid that the EDs send too many messages of this type.

\vspace{3mm}
\begin{figure}[H]
    \centering
    \includegraphics[width=0.7\linewidth]{images/pivot/NoJ.png}
    \caption{This plot reports an example of NoJ/weeks of a LoRaWAN network without anomaly devices. There is a peak only in the initial phase.}
    \label{fig:noj}
\end{figure}
\vspace{3mm}

\subsubsection{Number of Detected Devices (NoDD)}
The Number of Detected Devices (NoDD) describes the current amount of devices classified by PIVOT as vulnerable. These EDs, that re-joined the network at least once, have a too predictable pattern that could disclose their sensible characteristics, such as the identifiers, to external eavesdroppers. NoDD is updated each time PIVOT triggers the alert to the operator. This metric depends on the \textit{accuracy} of the detection algorithm of PIVOT, which may \textit{misclassify} some devices or miss capturing all vulnerable ones.

\subsubsection{Number of Unique Devices (NoUD)}
The Number of Unique Devices (NoUD) is the number of ED registered by PIVOT. These devices have sent at least one message to the server, exposing their DevAdress. NoUD does \textit{not} represent the total amount of devices of the network but only those that have transmitted packets since PIVOT was turned on. NoUD indicates the number of \textit{currently active} devices.

\subsubsection{Percentage of Detected Devices (PoDD)}
The Percentage of Detected Devices (PoDD) is a value in the range [0, 1] and represents the ratio between \textit{NoDD} and \textit{NoUD}. It provides a clear view of the network, showing exactly how many devices are vulnerable among all those present. Since it is not unusual for LoRa devices to generate periodic traffic, the PoDD value is generally always greater than zero. Using this metric, the operator can establish the \textit{maximum} permissible percentage of exposed devices, according to the dimension of the network. For example, in the case of environments made by a consistent number of devices, the management becomes more complex, and it is almost impossible to handle all the devices exposed. In this case, the goal could be to reduce the value of \textit{PoDD} whenever possible.