\chapter{Introduction}
% Challenges of today
\lettrine[lines=2, findent=3pt, nindent=0pt]{F}{}rom asset tracking and smart metering to air quality monitoring, industries make use of sensors to collect large amounts of data to improve efficiency and lower costs. Businesses face complex and remote environments where IoT applications require long-range connectivity, low data rate, and low energy consumption. Traditional technologies don't meet the coverage and power requirements needed \cite{MEKKI20191}. For example, Bluetooth is intended only for short-range uses and Wi-Fi is too expensive, power-hungry, and encounters significant coverage gaps. Furthermore, mesh networks, because of their excessively complex network, don't scale beyond medium-range applications.
\\
% LPWAN
Created for IoT applications that transmit small amounts of data, Low-Power Wide Area Network (LPWAN) is a class of wireless technologies with a range that varies from a few kilometers in urban areas to over 15 km in rural settings \cite{7721743}. LPWAN lightweight protocols reduce the complexity and costs of sensors, which, consequently, can operate for more than ten years, even if equipped with small batteries \cite{s20174794}. 
\\
% Introduzione su LoRaWAN
One prominent LPWAN modulation technique is LoRa. Developed by Cycleo of Grenoble and acquired by Semtech, LoRa works on the physical layer of the OSI model and enables a link for long-range communications. LoRaWAN is one of the various protocols created to define the upper layer of LoRa. It delineates the system architecture of the network and manages the communication frequencies, data rate, and power for all devices. Currently, 163 network operators in 177 countries \cite{coverage} adopt LoRaWAN, and its coverage is significantly expanding.

\section{Weaknesses in LoRaWAN}
The wide communication range of LoRaWAN causes the messages sent in the network can be intercepted by unauthorized eavesdroppers, even hundreds of meters away from endpoints. Despite LoRaWAN having built-in security mechanisms to protect the confidentiality of user data \cite{Moraes2021ASR}, such as encrypting the payload of messages exchanged, some relevant parameters remain exposed. Indeed, since the header of the packets is left-in-clear, potentially lifting metadata is shown, such as two LoRaWAN identifiers, the DevAddress and the DevEUI. The DevAddress is a temporary address generated by the network where devices are operating, while the DevEUI is a global, unique identifier from the manufacturer. Since the DevEUI represents a source of information about endpoints and their constructors, LoRaWAN's devices reveal it the least possible, preventing its association with other fields and, in particular, with the DevAddress. Indeed, endpoints expose their DevEUI only during the association phase, called OOTA, and show their current DevAddress only in uplink messages they send to the Network Server. Since cannot be a packet on the network with both of them exposed in the header, the connection between these two identifiers, known only by the network, is theoretically unlikely to externals. Nevertheless, several studies discovered several methodologies an attacker can exploit to connect these two elements. This weakness compromises the users' privacy and companies of the targeted network: by linking the DevEUI to the DevAddr, the intruder obtains further knowledge about the devices and their activities. It can discover associated applications, the manufacturer that produced them, the type of equipment, and their purposes.

\vspace{5mm} %5mm vertical space

% Introduzione a PIVOT
In this work, we present PIVOT (Privacy-Monitoring), a privacy-oriented system for LoRaWAN, conceived to preserve the identity of endpoints and data they exchanged in the network. PIVOT analyzes the LoRa RF traffic stream to identify devices exposed to privacy threats. Its detection procedure works with the pattern matching principles and runs using a real-time algorithm. In this way, PIOVT can notify the network in the shortest possible time about the presence of vulnerable devices. The measurements it outputs, such as the number of devices detected, can give the operator a view of the current safety state of the network, supporting it in applying immediate countermeasures.

\section{Structure of the thesis}
The rest of this thesis is structured as follows:
\begin{enumerate}
    \item Chapter \ref{lorawan} overviews the LoRaWAN technology and its features, focusing on the aspects of this protocol associated with the objectives of this thesis. 
    \item Chapter \ref{model} discusses security issues that concern LPWAN technologies and, in detail, the privacy-related vulnerabilities that affect the LoRaWAN protocol and their implications for the customers.
    \item Chapter \ref{pivot} introduces PIVOT, first defining the design goals, then explaining its core algorithm and the metrics applied. 
    \item In Chapter \ref{countermeasures} we explain the various countermeasures that the operator, based on the PIVOT output, can enforce to make the network securer.
    \item Chapter \ref{implementation} illustrates the implementation and the results of our experimental studies.
    \item Chapter \ref{conclusions} discusses conclusions and possible directions for future work.
\end{enumerate}