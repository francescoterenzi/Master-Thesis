LoRaWAN is a wireless technology developed to transmit over long distances using low power. It runs over the proprietary LoRa radio modulation and provides fundamental IoT requirements such as bi-directional communication, end-to-end security, mobility, and localization services. Despite LoRaWAN guarantees confidentiality and integrity of transmitted data, the wireless nature of the medium causes that an eavesdropper, listening to the network communications, can collect unencrypted elements stored in the packets. In particular, it can obtain two sensible metadata elements, called DevAddress and DevEUI. Since the association between these elements can involve privacy issues, LoRaWAN forces endpoints to expose their DevEUI only during the association procedure. In the first part of this work, we prove how an adversary, exploiting well-known vulnerabilities of LoRaWAN, can link them anyway. Then we explain the consequences for the privacy of devices and users that joined the network and propose PIVOT (Privacy-Monitoring), an analyzer system for LoRaWAN that detects in real-time vulnerable endpoints. Furthermore, we explain how the metrics used in PIVOT can support the operator in applying adequate countermeasures. Finally, we test our scheme on a simulated LoRaWAN application and examine the results obtained.