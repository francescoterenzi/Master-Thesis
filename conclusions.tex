\chapter{Conclusions}
\label{conclusions}
In this thesis, we have examined that although the LoRaWAN protocol is designed to be secure and reliable, it suffers from still unsolved privacy issues, which an unauthorized eavesdropper can exploit to obtain unauthorized information about devices and users of a LoRa network. In particular, by investigating the different existing attack techniques, we have concluded that by analyzing the behavior of the devices in the traffic, an attacker, contrary to what is stated in the LoRaWAN specification, can correctly identify with high accuracy the couples <DevAddr, DevEUI> of the endpoints. To preserve the confidentiality of the data transmitted even in presence of these traffic analysis techniques, we presented a preventive strategy named PIVOT, a privacy monitoring system to integrate into an exposed network, which, by analyzing the traffic generated by the endpoints, identify possible critical situations and possibly alert the network operator. In detail, PIVOT takes as input the LoRa flow, elaborates it in real-time through a \textit{detection} algorithm, and outputs different metrics about the status of the network. In the last part of the work, we introduced a Python simulator to test PIVOT and discussed the results obtained. In the evaluation procedure, we first examined the \textit{sensibility} of the system i.e. the ability to correctly recognize all possible matches when some variables of the flow change. Then, in a second test, comparing two LoRa flows, we demonstrate that if the operator alters the transmission frequencies of the EDs by adding delays between the inter-arrival times of packets, the percentage of detections by PIVOT drops to 97\%, effectively making the network less vulnerable.

\vspace{3mm}

The detection algorithm of PIVOT proves to be very accurate and able to recognize a high percentage of vulnerable devices, regardless of the number of active EDs in the network. On the contrary, it suffers slightly when the patterns to be identified become more complex. But it is an uncommon scenario in real LoRa networks, which by their nature, devices tend to transmit a small amount of data to guarantee maximum energy efficiency.

\vspace{3mm}

\section{Next steps}
PIVOT is only a starting point and offers significant opportunities for improvement. The detection algorithm, which is the heart of this system, performs a too strict pattern analysis, making the system not yet fully adaptable to more complex situations than those analyzed in this work, where devices may have unexpected transmission errors, sending for example abnormal packets, or may not necessarily follow any pattern. Furthermore, it uses a binary classification system that catalogs each DevAddress as belonging to a new device or a re-joined one. This scheme can be further refined by adding different levels of classification and alerting the operator only when the most critical one is reached. Finally, the \textit{update} subroutine, which systematically analyzes the patterns and verifies their completeness, can be improved. Indeed, the \texttt{verified} flag, which is currently a Boolean variable, can be replaced by a numeric value in the range [0,1], where the extremes 0 and 1 respectively indicate the minimum and maximum probability that the detected pattern conforms to the original one, allowing greater accuracy in filtering reliable patterns to use during the matching procedure. In this way, the value of the flag does not change unexpectedly in the presence of packets whose transmission was not expected but gradually decreases. Future researches of this work will focus on refining the detection algorithm, possibly integrating it with stream-processing frameworks to speed up some steps and make detection even faster, regardless of the number of devices or packets transmitted. Finally, the PIVOT system can be expanded by adding new features, and modules or adapting it for other LPWAN technologies.